% In this file you should put the actual content of the blueprint.
% It will be used both by the web and the print version.
% It should *not* include the \begin{document}
%
% If you want to split the blueprint content into several files then
% the current file can be a simple sequence of \input. Otherwise It
% can start with a \section or \chapter for instance.

A factorization system in a category $\mathcal{C}$ consists of two classes of morphisms $(L,R)$, such that both $L$ and $R$ contain isomorphisms and are closed under composition, and every morphism $f: C \to D$ in $\mathcal{C}$ admits a factorization into a morphism $l\in L$ followed by a morphism $r'\in R$, which is unique up to unique isomorphism among such factorizations.
\[\begin{tikzcd}
	&& E \\
	C &&&& D \\
	&& {E'}
	\arrow["r", from=1-3, to=2-5]
	\arrow["i"', from=1-3, to=3-3]
	\arrow["\cong", from=1-3, to=3-3]
	\arrow["l", from=2-1, to=1-3]
	\arrow["{l'}"', from=2-1, to=3-3]
	\arrow["{r'}"', from=3-3, to=2-5]
\end{tikzcd}\]
Plan of work:
\begin{itemize}
    \item show that factorization systems descend to slices
    \item construct a few elementary examples (in the \textbf{Set}, \textbf{Grp}, \textbf{CRing} etc.)
    \item show that the left class in a f.s. has the right cancellation property, and, dually, the right class has the left cancellation property
    \item define the orthogonality relation and show various closure properties of orthogonal complements (under (co)limits, (co)base change, etc.) 
    \item either of the classes in a f.s. determines the other via orthogonal complements
    \item recognition principle for factorization systems: $(L,R)$ is a factorization system iff every morphism $f: C \to D$ in $\mathcal{C}$ admits an (apriori non-unique) $(L,R)$-factorization, both classes are replete, and $L\bot R$
\end{itemize}
If time permits it, we would also like to use the general theory of factorization systems to define (left-exact) modalities and study some of their properties. In particular, we would like to show that every (left-exact) modality determines a (left-exact) reflective subcategory and vice versa.